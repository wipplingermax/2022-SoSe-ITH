\vorlesung{6}{Untentscheidbare Probleme}{10.05.2022}

\section*{4.1 Untentscheidbare Probleme}

\textbf{Ziel}: \\
Konstruktion von Problemen, die nicht durch TM entscheidbar sind.

\begin{defn}{diagonales Halteproblem}
    Die Menge $H_{diag} = \{e \in \mathbb{N}_0 : \Phi_e(e) \downarrow\}$ heißt das \textbf{diagonale Halteproblem}
\end{defn}

\begin{prop}{}
    Das diagonale Halteproblem ist rekursiv aufzählbar. \\
    
    \textbf{Beweis:} \\
    Die DTM, die bei eingabe $e \in \mathbb{N}_0$ arbeitet wie $U$ bei
    der Eingabe $(e,e)$, aber bei terminierter Simulation 1 ausgibt, statt der
    Ausgabe der Simulation, berechnet die partielle charakterischte Funktion von $H_{diag}$.
    Die partielle Funktion ist somit partiell berechenbar \\
    $\Rightarrow H_{diag}$ ist rekursiv aufzählbar $\square$ \\

    \textbf{Erinnerung:} \\
    Bem. 3.12: Eine Sprache $L$ ist genau dann entscheidbar, falls $L$ und $L^c$
    rekursiv aufzählbar sind. \\

    $\Rightarrow$ Unser Plan: Wir zeigen $H_{diag}$ ist NICHT rekursiv aufzählbar.
\end{prop}

\begin{satz}{Das diagonale Halteproblem ist entscheidbar}
    \textbf{Beweis:} \\
    Angenommen $H_{diag}$ wäre entscheidbar. \\
    Dann wäre die partielle charakteristische Funktion $\phi$ von $H_{diag}^c := \mathbb{N}_0 \backslash H_{diag}$
    partiell berechenbar. \\

    $\Rightarrow$ Es gibt einen Index $e$ von $\phi$. Es folgt $e \in H_{diag}^c$
    $\Leftrightarrow$ Es gibt einen Index $e$ von $\phi$. Es folgt \\
    $e \in H_{diag}^c \Leftrightarrow \phi(e)\downarrow = \Phi_e(e)\downarrow \Leftrightarrow e \in H_{diag} \Leftrightarrow e \notin H_{diag}^c \lightning \square$



\end{satz}